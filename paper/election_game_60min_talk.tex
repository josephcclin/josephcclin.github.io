\documentclass[xcolor=dvipsnames,envcountsect]{beamer}
%\documentclass[xcolor=dvipsnames]{beamer}
\usecolortheme[named=MidnightBlue]{structure}
%\usecolortheme{seahorse}
%\usetheme[height=8mm]{Rochester}
\usetheme{Madrid}
\useoutertheme{smoothtree} 
\setbeamertemplate{bibliography item}[text]
%\setbeamertemplate{navigation symbols}{} 
\setbeamertemplate{theorems}[numbered]
%\usepackage{beamerthemeshadow}
%\setbeamertemplate{footline}[frame number]{}
\setbeamertemplate{blocks}[rounded][shadow=true] 
%\def\urltilde{\kern -.15em\lower .7ex\hbox{\~{}}\kern .04em}
%\pgfdeclareimage[height=0.9cm]{institution-logo}{aslogo_w} 
%\logo{\pgfuseimage{institution-logo}}
%"ccu-logo" can be substituted by "ccu-tcs-logo"
\usepackage{amsmath,bm}
\usepackage{latexsym}
\usepackage{amssymb,bm}
\usepackage{graphicx}
\usepackage{amsthm}
\usepackage{amsfonts}
\usepackage{boxedminipage}
\usepackage{mathrsfs}
\usepackage{hyperref}
\usepackage{comment}
\usepackage{marvosym}
\newtheorem{claim}{Claim}
\newtheorem*{question}{Question}
\newtheorem*{observation}{Observation}
\newtheorem*{theory}{Theorem}
\newtheorem*{Euler}{Euler's formula}
\newtheorem*{prop}{Proposition}
\newtheorem*{property}{Property} 
\newtheorem*{defn}{Definition}
\newtheorem*{lemm}{Lemma}
\newtheorem*{conj}{Conjecture}
%\newcommand{\dotcup}{\ensuremath{\mathaccent\cdot\cup}}
\renewcommand{\figurename}{\textbf{Fig.}}


\title[Two Party Election Game]{How Good is a Two-Party Election Game?}
\author[Lin, Lu, Chen]{
Speaker: Chuang-Chieh Lin\vspace{9pt}\\
Joint work with \vspace{9pt}\\
Chi-Jen Lu and Po-An Chen
} 

\institute[TKU, IIS AS, NCYU]{
}
\date[17 Jun 2021]{
{\footnotesize Invited Talk in National Taipei University of Business}\vspace{7pt}\\
{\small 17th June 2021}}


\begin{document}
\maketitle
%\beamertemplatetransparentcovereddynamic

\begin{frame}
\vspace{12pt}
\begin{figure}
	\begin{center}
		\includegraphics[scale=0.45]{eps/authors.eps}
	\end{center}
\end{figure}	
\end{frame}


\begin{frame}
\frametitle{Outline}
\tableofcontents%[pausesections]
\end{frame}

%%%%%%%%%%%%%%%%%%%%%%%%%%%%%%%%%%%%%%%%%%%%%%%%%%%%%%%%%%%%%%%%%%
\section{Introduction and Motivations}
%%%%%%%%%%%%%%%%%%%%%%%%%%%%%%%%%%%%%%%%%%%%%%%%%%%%%%%%%%%%%%%%%%


\begin{frame}
\frametitle{Outline}
\tableofcontents[currentsection,currentsubsection]
\end{frame}


\begin{frame}
\frametitle{The Inspiration}
\begin{figure}
	\begin{center}
		\includegraphics[scale=0.23]{eps/Lincoln.eps}
	\end{center}
\end{figure}
\begin{quotation}
\noindent {\small ``[\ldots] and that government of the people, by the people, for the people, shall not perish from the earth." \\
	\hfill--- Abraham Lincoln, 1863.}
\end{quotation}

\end{frame}

\begin{frame}
\frametitle{Motivations (I): Why The Two-Party System?}
\vspace{-6pt}
\begin{figure}
	\begin{center}
		\includegraphics[scale=0.30]{eps/Duverger.eps}
	\end{center}
\end{figure}
\begin{quotation}
\noindent {\small ``The simple-majority single-ballot system favours the two-party system."
\hfill--- Maurice Duverger, 1964.}
\end{quotation}
\end{frame}


\begin{frame}{Motivations (II): Social Choice Rules}
\quad Example: 
\begin{itemize}
	\item Each voter provides an ordinal ranking of the candidates, 
	\item Aggregate these rankings to produce either a single winner or a consensus ranking of all (or some) candidates. 
\end{itemize}
\pause
	\begin{block}{Gibbard--Satterthwaite Theorem (1973)}
	Given a deterministic electoral system that choose a single winner. For every voting rule, one of the following three things must hold:
	\begin{itemize}
		\item The rule is dictatorial. 
		\item The rule limits the possible outcomes to two alternatives only. 
		\item The rule is susceptible to tactical voting. 
	\end{itemize}
	\end{block}
\end{frame}


\begin{frame}{Motivations (III): Distortion of Social Choice Rules}
\begin{figure}
	\begin{center}
		\includegraphics[scale=0.33]{eps/OfThePeopleDistortion_01.eps}
	\end{center}
\end{figure}
\pause
\begin{itemize}
	\item The average distance from the population to candidate L: $\approx 0.5$.  
	\vspace{3pt}
	\item The average distance from the population to candidate R: $\approx 1.5$. 
	\vspace{3pt}
	\item But R will be elected as the winner in the election. 
\end{itemize}    
\end{frame}


\begin{frame}
\frametitle{Issues of Previous Studies}
\begin{itemize}
    \item Voters' behavior on a \textcolor{OliveGreen}{micro-level}. 
    \begin{itemize}
        \item Voters are strategic; 
        \item Voters have different preferences for the candidates.
        \item Various election rules result in different winner(s).
        \pause
        \item [] $\vdots$
    \end{itemize}
\end{itemize}
\end{frame}


\begin{frame}
\frametitle{Our Focus}
\begin{itemize}
    \item We consider an intuitive \textcolor{blue}{macro} perspective instead. 
    \begin{itemize}
        \item Parties are players; 
        \item The strategies can be their nominated candidates (or policies); 
        \pause
        \item The point is: \pause 
            \begin{itemize}
                \item Who is \alert{more likely to win} the election campaign and \alert{how likely} is it?
                \pause
                \item Is the game \alert{stable} in some sense?
                \pause
                \item What's the \alert{price for stability} which resembles ``the distortion"?
            \end{itemize}
    \end{itemize}
\end{itemize}

\end{frame}


\begin{frame}
\begin{figure}
	\begin{center}
		\includegraphics<1>[scale=0.45]{eps/2party_candidates.eps}
		\includegraphics<2>[scale=0.45]{eps/2party_social_utilities.eps}
	\end{center}
\end{figure}
\end{frame}


\iffalse
\begin{frame}
\begin{figure}
	\begin{center}
		\includegraphics<2>[scale=0.45]{eps/2party_social_utilities.eps}
	\end{center}
\end{figure}
\end{frame}
\fi


\begin{frame}
\begin{figure}
	\begin{center}
		\includegraphics<1>[scale=0.45]{eps/2party_social_utilities_deviate12.eps}
		\includegraphics<2>[scale=0.45]{eps/2party_social_utilities_deviate22.eps}
		\includegraphics<3>[scale=0.45]{eps/2party_social_utilities_deviate21.eps}
		\includegraphics<4>[scale=0.45]{eps/2party_social_utilities_deviate11.eps}
	\end{center}
\end{figure}
\end{frame}


\iffalse
\begin{frame}
\begin{figure}
	\begin{center}
		\includegraphics[scale=0.45]{eps/2party_social_utilities_deviate22.eps}
	\end{center}
\end{figure}
\end{frame}


\begin{frame}
\begin{figure}
	\begin{center}
		\includegraphics[scale=0.45]{eps/2party_social_utilities_deviate21.eps}
	\end{center}
\end{figure}
\end{frame}


\begin{frame}
\begin{figure}
	\begin{center}
		\includegraphics[scale=0.45]{eps/2party_social_utilities_deviate11.eps}
	\end{center}
\end{figure}
\end{frame}
\fi


\begin{frame}{Concept of Stability: Pure Nash Equilibrium}
\begin{itemize}
	\item Each party's strategy: candidate nomination.
	\item \alert{Pure Nash equilibrium (PNE)}: Neither party $A$ nor $B$ wants to deviate (i.e., change) from their strategy (i.e., nomination) unilaterally.
\end{itemize}
\end{frame}


\begin{frame}
	\quad {\small An instance with a PNE.}
	\begin{figure}
		\begin{center}
			\includegraphics[scale=0.37]{eps/2party_deviation_NE_01.eps}
		\end{center}
	\end{figure}
\end{frame}


\begin{frame}
	\quad {\small An instance with a PNE (expected social utility: $8.55$).}
	\begin{figure}
		\begin{center}
			\includegraphics[scale=0.37]{eps/2party_deviation_NE_02.eps}
		\end{center}
	\end{figure}
\end{frame}


\begin{frame}{A Kind of Inefficiency Measure: The Price of Anarchy}
	\quad {\small An instance with a PNE (expected social utility: $8.55$, optimum: $9$).}
	\begin{figure}
		\begin{center}
			\includegraphics[scale=0.37]{eps/2party_deviation_NE_03.eps}
		\end{center}
	\end{figure}
\vspace{-7pt}
\begin{itemize}
	\begin{small}
	\item The \alert{price of anarchy (POA)}: $\frac{9}{8.55} \approx 1.05$. 
	\end{small}
\end{itemize}
\end{frame}



%%%%%%%%%%%%%%%%%%%%%%%%%%%%%%%%%%%%%%%%%%%%%%%%%%%%%%%%%%%%%%%%%%
\section{The Formal Setting}
%%%%%%%%%%%%%%%%%%%%%%%%%%%%%%%%%%%%%%%%%%%%%%%%%%%%%%%%%%%%%%%%%%


\begin{frame}
\frametitle{Outline}
\tableofcontents[currentsection,currentsubsection]
\end{frame}


\begin{frame}
\frametitle{Two-Party Election Game: Formal Setting}
\begin{itemize}
\item Party $A$: $m$ candidates $A_1,A_2,\ldots,A_m$. \\Party $B$: $n$ candidates $B_1,B_2,\ldots,B_n$.
\vspace{8pt}
\item $A_i$: brings utility $u(A_i) = u_A(A_i) + u_B(A_i) \in [0,b]$, \\$B_j$: brings utility $u(B_j) = u_A(B_j) + u_B(B_j) \in [0,b]$, for some $b\geq 1$. 
	\begin{itemize}
		\item $u_A(A_1)\geq u_A(A_2)\geq \ldots \geq u_A(A_m)$,  $u_B(B_1)\geq u_B(B_2)\geq \ldots \geq u_B(B_n)$
	\end{itemize}
\vspace{5pt}
\item $p_{i,j}$: $\Pr[A_i \mbox{ wins over } B_j]$. 
\vspace{5pt}
\item Expected utilities:
\vspace{-7pt}
\begin{eqnarray*}
	a_{i,j} &=& p_{i,j}u_A(A_i) + (1-p_{i,j})u_A(B_j)\\
	b_{i,j} &=& (1-p_{i,j})u_B(B_j) + p_{i,j}u_B(A_i).
\end{eqnarray*} 
\end{itemize}
\end{frame}


\begin{frame}
	\frametitle{Egoism (Selfishness)}
	\begin{figure}
		\begin{center}
			\includegraphics[scale=0.45]{eps/2party_egoism.eps}
		\end{center}
	\end{figure}
\end{frame}


\begin{frame}
	\frametitle{Two-Party Election Game: Formal Setting (contd.)}
	\begin{itemize}
		\item Party $A$: $m$ candidates $A_1,A_2,\ldots,A_m$. \\Party $B$: $n$ candidates $B_1,B_2,\ldots,B_n$.
		\vspace{8pt}
		\item $A_i$: brings utility $u(A_i) = u_A(A_i) + u_B(A_i) \in [0,b]$,  \\$B_j$: brings utility $u(B_j) = u_A(B_j) + u_B(B_j) \in [0,b]$, for some $b\geq 1$. 
		\begin{itemize}
			\item $u_A(A_1)\geq u_A(A_2)\geq \ldots \geq u_A(A_m)$,  $u_B(B_1)\geq u_B(B_2)\geq \ldots \geq u_B(B_n)$
		\end{itemize}
		\vspace{5pt}
		\item $p_{i,j}$: $\Pr[A_i \mbox{ wins over } B_j]$. 
		\vspace{5pt}
		\item Expected utilities:
		\vspace{-7pt}
		\begin{eqnarray*}
			a_{i,j} &=& p_{i,j}u_A(A_i) + (1-p_{i,j})u_A(B_j)\\
			b_{i,j} &=& (1-p_{i,j})u_B(B_j) + p_{i,j}u_B(A_i).
		\end{eqnarray*} 
		\item \structure{egoistic}: $u_A(A_i)> u_A(B_j)$ and $u_B(B_j)> u_B(A_i)$ for all $i\in [m], j\in [n]$.
	\end{itemize}
\end{frame}


\begin{frame}
	\frametitle{Two-Party Election Game: Formal Setting (contd.)}
	\begin{itemize}
		\item Three models on $p_{i,j}$: 
		\vspace{5pt}
		\begin{itemize}
			\item \alert{Bradley-Terry (Na\"{i}ve)}: $p_{i,j} := u(A_i)/(u(A_i) + u(B_j))$
			\begin{itemize}
				\item \textcolor{blue}{Linear} dependency on the two social utilities.
				\item Intuitive. 
			\end{itemize}
			\vspace{3pt}
			\item \alert{Linear link}: $p_{i,j} := (1 + (u(A_i) - u(B_j))/b)/2$.
			\begin{itemize}
				\item \textcolor{blue}{Linear} on the  \textcolor{blue}{difference} between the two social utilities.
				\item Dueling bandit setting.
			\end{itemize}
			\vspace{3pt}
			\item \alert{Softmax}: $p_{i,j} := e^{u(A_i)/b}/(e^{u(A_i)/b} + e^{u(B_j)/b})$
			\begin{itemize}
				\item Bivariate \textcolor{blue}{nonlinear} rational function of the two social utilities. 
				\item Extensively used in machine learning. 
			\end{itemize}
		\end{itemize}
	\end{itemize}
\end{frame}



\begin{frame}
	\frametitle{Two-Party Election Game: Formal Setting (contd.)}
	\begin{itemize}
		\item The \structure{social welfare} of state $(i,j)$: 
		\begin{itemize}
			\item [] $SU_{i,j} = a_{i,j} + b_{i,j}$.
		\end{itemize}
		\vspace{6pt}
		\item $(i,j)$ is a \structure{PNE} if $a_{i',j}\leq a_{i,j}$ for any $i'\neq i$ 
		and $b_{i,j'} \leq b_{i,j}$ for any $j'\neq j$.
		\pause
		\vspace{6pt}
		\item The \structure{PoA} of the game: 
			\[
			\frac{SU_{i^*,j^*}}{SU_{\hat{i},\hat{j}}} = 
			\frac{a_{i^*,j^*} + b_{i^*,j^*}}{a_{\hat{i},\hat{j}}+b_{\hat{i},\hat{j}}},
			\]
		\begin{itemize}
			\item $(i^*,j^*) = \arg\max_{(i,j)\in [m]\times [n]}(a_{i,j}+b_{i,j})$: {\bf the optimal state}.
			\item $(\hat{i},\hat{j}) = \arg\min_{(i,j)\in [m]\times [n]\atop (i,j)\mbox{\scriptsize \;is a PNE}}(a_{i,j}+b_{i,j})$: the PNE with {\bf the worst} social welfare.
		\end{itemize}
	\end{itemize}
\end{frame}


%%%%%%%%%%%%%%%%%%%%%%%%%%%%%%%%%%%%%%%%%%%%%%%%%%%%%%%%%%%%%%%%%%
\section{The First Equilibrium Existence Results}
%%%%%%%%%%%%%%%%%%%%%%%%%%%%%%%%%%%%%%%%%%%%%%%%%%%%%%%%%%%%%%%%%%


\begin{frame}
	\frametitle{Outline}
	\tableofcontents[currentsection,currentsubsection]
\end{frame}


\begin{frame}
\frametitle{Preliminary Inspections for the PNE}
\quad Focus on $m=n=2$ first.
\vspace{8pt}
	\begin{itemize}
		\item First try: by human brains and human eyes. 
		\pause
		\begin{itemize}
			\item Difficult. \Frowny
		\end{itemize}
		\vspace{7pt}
		\pause
		\item Random sampling: \Smiley
		\begin{itemize}
			\item Sampling the values of~$u_A(A_i), u_B(A_i), u_A(B_j), u_B(B_j)$ for each $i, j$ and the constant $b$ for hundreds of millions times.
			\item Experiments for the three winning probability models.
		\end{itemize}
	\end{itemize}
\end{frame}


\begin{frame}{Example: No PNE in the Bradley-Terry Model}
	\quad $m = n = 2$, $b=100$ (left: egoistic, right: non-egoistic).
	\begin{footnotesize}
	\begin{table}[ht]
		\begin{center}
			\begin{tabular}[c]{ l l | l l }
				%\centering
				\multicolumn{4}{ c }{}\\
				$A$ & \multicolumn{1}{c}{}& $B$ & \\
				\hline
				$u_A(A_i)$ & $u_B(A_i)$ & $u_B(B_j)$ & $u_A(B_j)$\\
				\hline
				91  &  0  &  11  &  1\\
				90  &  8  &  10  &  20\\
				\hline
			\end{tabular} \;\;\;\;
			\begin{tabular}[c]{ l l | l l }
				%\centering
				\multicolumn{4}{ c }{}\\
				$A$ & \multicolumn{1}{c}{}& $B$ & \\
				\hline
				$u_A(A_i)$ & $u_B(A_i)$ & $u_B(B_j)$ & $u_A(B_j)$\\
				\hline
				44  &  10  &  37  &  17\\
				39  &  55  &  10  &  5\\
				\hline
			\end{tabular}
			\vspace{7pt}\\
			\begin{tabular}[c]{  l | l | l }
				%\multicolumn{2}{c}{} \vspace{7pt}\\
				\centering
				% & \\
				%\hline
				&$B_1$  &  $B_2$\\
				\hline
				$A_1$&80.51, 1.28  &  73.84, 2.17\\
				\hline
				$A_2$&80.29, 8.32  &  74.02, 8.23\\
				%\hline
			\end{tabular}
			\;\;
			\begin{tabular}[c]{  l | l | l }
				%\multicolumn{2}{c}{} \vspace{7pt}\\
				\centering
				% & \\
				%\hline
				&$B_1$  &  $B_2$\\
				\hline
				$A_1$&30.50, 23.50  &  35.52, 10.00\\
				\hline
				$A_2$&30.97, 48.43  &  34.32, 48.81\\
				%\hline
			\end{tabular}
		\end{center}
	\end{table}
	\end{footnotesize}
\end{frame}


\begin{frame}
\frametitle{\large Example: No PNE in the Linear-Link Model (Non-Egoism)}
\quad $m = n = 2$, $b=100$.
\begin{table}[ht]
	\begin{small}
	\begin{center}
		\begin{tabular}[c]{ l l | l l }
			%\centering
			\multicolumn{4}{ c }{}\\
			$A$ & \multicolumn{1}{c}{}& $B$ & \\
			\hline
			$u_A(A_i)$ & $u_B(A_i)$ & $u_B(B_j)$ & $u_A(B_j)$\\
			\hline
			50  &  10  &  10  &  90\\
			5   &  20  &  5   &  20\\
			\hline
		\end{tabular}
		\vspace{7pt}\\
		\begin{tabular}[c]{ l | l l | l l }
			%\multicolumn{2}{c}{} \vspace{7pt}\\
			\centering
			% & \\
			%\hline
			&$B_1$&&$B_2$\\
			\hline
			$A_1$&78, & 10  &  40.25, & 8.375\\
			\hline
			$A_2$&79.375, & 11.25  &  12.5, & 12.5\\
			%\hline
		\end{tabular}
	\end{center}
	\end{small}
\end{table}
\end{frame}


\begin{frame}
	\frametitle{Non-Egoistic Games Seem to Be Bad \Frowny}
	\begin{itemize}
		\item [$\star$] In our experiments, {\bf EVERY} \alert{egoistic} game instance in the \alert{linear-link/softmax} model has a PNE! 
		\vspace{8pt}
		\pause
		\item The following discussions on equilibrium existence consider only egoistic games. 
	\end{itemize}
\end{frame}


\begin{frame}
	\frametitle{The Dominating-Strategy Equilibrium}
	\begin{lemm}[The Dominating-Strategy Equilibrium]
		\begin{itemize}
		\item If $u(A_1) > u(A_i)$ for each $i\in [n]\setminus\{1\}$, then $(1,j^{\#})$ is a PNE for $j^{\#} = \arg\max_{j\in [m]} b_{1,j}$. 
		\item If $u(B_1) > u(B_j)$ for each $j\in [m]\setminus \{1\}$, then $(i^{\#}, 1)$ is a PNE for $i^{\#} = \arg\max_{i\in [n]} a_{i,1}$. 
		\end{itemize}
	\end{lemm}
	\pause
\begin{itemize}
	\item Hence, the puzzles come from the other cases...
\end{itemize}
\end{frame}


\begin{frame}
\frametitle{No PNE $\Leftrightarrow$ Cycles of Deviations}
\begin{figure}
	\begin{center}
		\includegraphics[scale=0.40]{eps/deviation_NoPNE.eps}
	\end{center}
\end{figure}
\end{frame}


\begin{frame}
\frametitle{Deviations $\rightarrow$ Inequalities}
\begin{columns}
\begin{column}[t]{6cm}
\begin{scriptsize}
\begin{eqnarray*}
	\Delta(D_1) &=& -\Delta(D'_1) = a_{2,1}-a_{1,1}\\
	&=&p_{2,1}u_A(A_2)+(1-p_{2,1})u_A(B_1)\\
	&&-(p_{1,1}u_A(A_1)+(1-p_{1,1})u_A(B_1))\\
	&=& -p_{1,1}(u_A(A_1)-u_A(A_2))\\ 
	&&+ (p_{2,1}-p_{1,1})(u_A(A_2)-u_A(B_1)).
\end{eqnarray*}
\begin{eqnarray*}
	\Delta(D_2) &=& -\Delta(D'_2) = b_{2,2}-b_{2,1}\\
	&=&(1-p_{2,2})u_B(B_2)+p_{2,2}u_B(A_2)\\ 
	&&- ((1-p_{2,1})u_B(B_1)+p_{2,1}u_B(A_2))\\
	&=& -(1-p_{2,1})(u_B(B_1)-u_B(B_2))\\
	&&+(p_{2,1}-p_{2,2})(u_B(B_2) - u_B(A_2)).
\end{eqnarray*}
\end{scriptsize}
\end{column}
\begin{column}[t]{6cm}
\begin{scriptsize}
\begin{eqnarray*}
	\Delta(D_3) &=& -\Delta(D'_3) = a_{1,2}-a_{2,2}\\ &=&p_{1,2}u_A(A_1)+(1-p_{1,2})u_A(B_2)\\
	&&-(p_{2,2}u_A(A_2)+(1-p_{2,2})u_A(B_2))\\
	&=& p_{1,2}(u_A(A_1)-u_A(A_2))\\ 
	&&+ (p_{1,2}-p_{2,2})(u_A(A_2)-u_A(B_2)).
\end{eqnarray*}
\begin{eqnarray*}
	\Delta(D_4) &=& -\Delta(D'_4) = b_{1,1}-b_{1,2}\\ &=&(1-p_{1,1})u_B(B_1)+p_{1,1}u_B(A_1)\\ 
	&&- ((1-p_{1,2})u_B(B_2)+p_{1,2}u_B(A_1))\\
	&=& (1-p_{1,1})(u_B(B_1)-u_B(B_2))\\
	&&+(p_{1,2}-p_{1,1})(u_B(B_2) - u_B(A_1)).
\end{eqnarray*}
\end{scriptsize}
\end{column}
\end{columns}
\end{frame}


\begin{frame}
	\frametitle{The Crucial Lemma}
	\vspace{-7pt}
	\begin{figure}
		\begin{center}
			\includegraphics[scale=0.36]{eps/deviation_lemma.eps}
		\end{center}
	\end{figure}
\end{frame}


\begin{frame}
	\frametitle{The Crucial Lemma}
	\vspace{-5pt}
	\begin{lemm}[Main Lemma for the Linear-Link \& Softmax Models]
	\label{lem:devConf_LL}
	Consider the two-party election game in the linear-link/softmax model. 
	\begin{itemize}
	 \item If $u(A_2) > u(A_1)$, then 
	 	\begin{itemize}
	 		\item $\Delta(D_2) > 0\Rightarrow \Delta(D_4) < 0$ 
	 		\item $\Delta(D_4) > 0\Rightarrow \Delta(D_2) < 0$. 
	 	\end{itemize}
 	\item If $u(B_2) > u(B_1)$, then 
 		\begin{itemize}
 			\item $\Delta(D'_1) > 0 \Rightarrow \Delta(D'_3) < 0$.
 			\item $\Delta(D'_3) > 0 \Rightarrow \Delta(D'_1) < 0$.
 		\end{itemize}
 	\end{itemize}
	\end{lemm}
\pause
\begin{theory}[First Equilibrium Existence Result for $m=n=2$]
	In the linear-link/softmax model with $m = n = 2$, the two-party election game always has a PNE.  \Smiley
\end{theory}
\end{frame}


%%%%%%%%%%%%%%%%%%%%%%%%%%%%%%%%%%%%%%%%%%%%%%%%%%%%%%%%%%%%%%%%%%%%%
\section{Generalization: $\geq 2$ Candidates for Each Party}
%%%%%%%%%%%%%%%%%%%%%%%%%%%%%%%%%%%%%%%%%%%%%%%%%%%%%%%%%%%%%%%%%%%%%


\begin{frame}
\frametitle{Outline}
\tableofcontents[currentsection,currentsubsection]
\end{frame}



\begin{frame}
\frametitle{What if a party has three or more candidates?}
\begin{figure}
	\begin{center}
		\includegraphics[scale=0.40]{eps/generalNE.eps}
	\end{center}
\end{figure}
\pause
\begin{theory}[Equilibrium Existence Result for $m,n\geq 2$]
	The two-party election game with $m\geq 2$ and $n\geq 2$ always has a PNE in the linear-link/softmax model. \Smiley
\end{theory}
\end{frame}


\begin{frame}
\frametitle{Summary of Our Results}
\begin{table}[ht]
\begin{center}
	\renewcommand{\arraystretch}{1.25}
	\begin{tabular}{ r|c|c|c| }
		 \multicolumn{1}{r}{}
		&  \multicolumn{1}{r}{Linear Link}
		&  \multicolumn{1}{r}{Bradley-Terry}
		&  \multicolumn{1}{r}{Softmax}\\
		\cline{2-4}
		PNE w/ egoism & \checkmark & $\times$ & \checkmark \\
		%\cline{2-4}
		PNE w/o egoism & $\times$ & $\times$ & ?$^{\#}$  \\
		\cline{2-4}
		%\hline
		%\hline
		%Worst PoA w/ egoism & $\leq 2^*$ & $\leq 2$ & $\leq 1+e$ \\
		%Worst PoA w/o egoism & $\infty$ & $\infty$ & $\infty$ \\
		%\cline{2-4}
	\end{tabular}
	\vspace{7pt}
\label{tab:summary}
\end{center}
\end{table}

\end{frame}


%%%%%%%%%%%%%%%%%%%%%%%%%%%%%%%%%%%%%%%%%%%%%%%%%%%%%%%%%%%%%%%%%%%%%
\section{The Price of Anarchy Bounds}
%%%%%%%%%%%%%%%%%%%%%%%%%%%%%%%%%%%%%%%%%%%%%%%%%%%%%%%%%%%%%%%%%%%%%


\begin{frame}
\frametitle{Outline}
\tableofcontents[currentsection,currentsubsection]
\end{frame}


\begin{frame}
\frametitle{Relating PNE to OPT}
\begin{itemize}
	\item $i$ dominates $i'$:  $i < i'$ and $u(A_i) > u(A_{i'})$.
\end{itemize}
\begin{lemm}[Property I: PNE and Domination]
	\begin{itemize}
		\item \exists $i'$, \; $i'$\; dominates \; $i$ \; $\Rightarrow$ $(i,j)$ is not a PNE for any $j\in [n]$.
		\item \exists $j'$, \; $j'$\; dominates \; $j$ \; $\Rightarrow$ $(i,j)$ is not a PNE for any $i\in [m]$.
	\end{itemize}
\end{lemm}	
	
\begin{prop}[Property II: Relating a PNE to the OPT State]
	Let's say we have  
	\begin{itemize}
		\item $(i,j)$: a PNE 
		\item $(i^*,j^*)$: the optimal state. 
	\end{itemize}
	Then, $u(A_i)+u(B_j)\geq \max\{u(A_{i^*}), u(B_{j^*})\}$.
\end{prop}
\end{frame}


\begin{frame}
\frametitle{Illustrating Example: In the Linear-Link Model}
For $i\in [m]$, $j\in [n]$,
\begin{small}
\begin{eqnarray*}
	SU_{i,j} &=& p_{i,j}\cdot u(A_i) + (1-p_{i,j})\cdot u(B_j)\\
	&=&\frac{1+(u(A_i)-u(B_j))/b}{2} \cdot u(A_i)
	+ \frac{1-(u(A_i)-u(B_j))/b}{2}\cdot u(B_j)\\
	&=& \frac{1}{2}(u(A_i)+u(B_j)) + \frac{1}{2b}(u(A_i)-u(B_j))^2\\
	&\geq& \frac{1}{2}(u(A_i)+u(B_j)).
\end{eqnarray*}
\end{small}
and 
\begin{small}
\begin{eqnarray*}
	SU_{i,j} &=& p_{i,j}\cdot u(A_i) + (1-p_{i,j})\cdot u(B_j)\leq \max\{u(A_i), u(B_j)\}.
\end{eqnarray*}
\end{small}

\end{frame}


\begin{frame}
\frametitle{Illustrating Example: In the Linear-Link Model (contd.)}
\begin{theory}[PoA Bound for Linear-Link]
	The two-party election game in the linear link model 
	has PoA $\leq~2$. 
\end{theory}
\begin{proof}
	$(i,j)$: a PNE; $(i^*,j^*)$: OPT. By the previous Lemma: 
	\[
	\left\{\begin{array}{l}
		i \mbox{ is not dominated by } i^*\\
		j \mbox{ is not dominated by } j^*
	\end{array}
	\right.
	\Rightarrow
	\left\{\begin{array}{l}
		i\leq i^*\mbox{ or } u(A_{i^*})\leq u(A_i)\\
		j\leq j^*\mbox{ or } u(B_{j^*})\leq u(B_j)
	\end{array}
	\right.
	\]
	\vspace{-7pt}
	\begin{itemize}
	\item $SU_{i^*,j^*}\leq \max\{u(A_{i^*}), u(B_{j^*})\}$,  $\max\{u(A_{i^*}), u(B_{j^*})\}\leq u(A_i)+u(B_j)$.
	\item $2\cdot SU_{i,j}\geq u(A_i)+u(B_j)$.
	\end{itemize} 
	Thus, $SU_{i,j}\geq SU_{i^*,j^*}/2$. 
	\qed
\end{proof}
\end{frame}


\begin{frame}
\frametitle{\large Illustrating Example: In the Linear-Link Model (Lower Bound)}
\begin{itemize}
	\item A tight example (PoA $\approx 2$; $\delta \ll \epsilon\ll b$).
\end{itemize}
\begin{table}[ht]
	\begin{center}
		\begin{tabular}[c]{ l l | l l }
			%\centering
			\multicolumn{4}{ c }{}\\
			$A$ & \multicolumn{1}{c}{}& $B$ & \\
			\hline
			$u_A(A_i)$ & $u_B(A_i)$ & $u_B(B_j)$ & $u_A(B_j)$\\
			\hline
			$\epsilon$  &  0 &  $\epsilon$  &  0\\
			$\epsilon-\delta$   &  $\epsilon-\delta$  &  $\epsilon-\delta$   &  $\epsilon-\delta$\\
			\hline
		\end{tabular}
		\vspace{12pt}\\
		\begin{tabular}[c]{  l  | l l | l l}
			\centering
			% & \\
			%\hline
			&$B_1$&&$B_2$\\
			\hline 
			$A_1$&$\frac{\epsilon}{2}$, & $\frac{\epsilon}{2}$  &  
			$\epsilon-\frac{\delta}{2}$, & $\frac{\epsilon}{2}-\frac{\delta}{2}$\\[2pt]
			\hline 
			$A_2$&$\frac{\epsilon}{2}-\frac{\delta}{2}$, &  $\epsilon-\frac{\delta}{2}$  &  
			$\epsilon-\delta$, & $\epsilon-\delta$\\
			%\hline
		\end{tabular}
	\end{center}
\end{table}

\end{frame}


\begin{frame}
\begin{center}
	{\Large The PoA of non-egoistic games can be really bad...}
\end{center}
\end{frame}


\begin{frame}
	\frametitle{Unbounded PoA for Non-Egoistic Games}
	\quad Linear-Link Model:
	\begin{table}[ht]
		\begin{center}
			\begin{tabular}[c]{ l l | l l }
				%\centering
				\multicolumn{4}{ c }{}\\
				$A$ & \multicolumn{1}{c}{}& $B$ & \\
				\hline
				$u_A(A_i)$ & $u_B(A_i)$ & $u_B(B_j)$ & $u_A(B_j)$\\
				\hline
				$\epsilon$  &  0  &  $\epsilon$  &  0\\
				0  &  $b$  &  0  &  $b$\\
				\hline
			\end{tabular}
			\vspace{10pt}\\
			\begin{tabular}[c]{c | c c | c c }
				%\multicolumn{2}{c}{} \vspace{7pt}\\
				\centering
				% & \\
				%\hline
				&$B_1$&&$B_2$\\
				\hline
				$A_1$&$\frac{\epsilon}{2}$, & $\frac{\epsilon}{2}$  &  
				$b-\frac{\epsilon(b-\epsilon)}{2b}$, & 0\\[2pt]
				\hline
				$A_2$&0, & $b-\frac{\epsilon(b-\epsilon)}{2b}$  &  $\frac{b}{2}$, & $\frac{b}{2}$\\[3pt]
				%\hline
			\end{tabular}
		\end{center}	
	\end{table}
	\begin{itemize}
		\item PoA $= \frac{b}{\epsilon}$.
	\end{itemize}
\end{frame}


\begin{frame}
	\frametitle{Unbounded PoA for Non-Egoistic Games}
	\quad Softmax Model:
	\begin{table}[ht]
		\begin{center}
			\begin{tabular}[c]{ l l | l l }
				%\centering
				\multicolumn{4}{ c }{}\\
				$A$ & \multicolumn{1}{c}{}& $B$ & \\
				\hline
				$u_A(A_i)$ & $u_B(A_i)$ & $u_B(B_j)$ & $u_A(B_j)$\\
				\hline
				$\epsilon$  &  0  &  $\epsilon$  &  0\\
				0  &  $b$  &  0  &  $b$\\
				\hline
			\end{tabular}
			\vspace{10pt}\\
			\begin{tabular}[c]{c | c c | c c }
				%\multicolumn{2}{c}{} \vspace{7pt}\\
				\centering
				% & \\
				%\hline
				&$B_1$&&$B_2$\\
				\hline
				$A_1$&$\frac{\epsilon e^{\epsilon}}{e^{\epsilon}+1}$, & $\frac{\epsilon e^{\epsilon}}{e^{\epsilon}+1}$  &  
				$\frac{\epsilon e^{\epsilon}+eb}{e^{\epsilon}+1}$, & 0\\[2pt]
				\hline
				$A_2$&0, & $\frac{\epsilon e^{\epsilon}+eb}{e^{\epsilon}+1}$  &  $\frac{b}{2}$, & $\frac{b}{2}$\\[3pt]
				%\hline
			\end{tabular}
		\end{center}	
	\end{table}
	\begin{itemize}
		\item PoA $= \frac{b}{2\epsilon e^{\epsilon}/(e^{\epsilon}+1)}$.
	\end{itemize}
\end{frame}


\begin{frame}
	\frametitle{Unbounded PoA for Non-Egoistic Games}
	\quad Bradley-Terry Model:
	\begin{table}[ht]
		\begin{center}
			\begin{tabular}[c]{ l l | l l }
				%\centering
				\multicolumn{4}{ c }{}\\
				$A$ & \multicolumn{1}{c}{}& $B$ & \\
				\hline
				$u_A(A_i)$ & $u_B(A_i)$ & $u_B(B_j)$ & $u_A(B_j)$\\
				\hline
				$\epsilon$  &  0  &  $\epsilon$  &  0\\
				0  &  $b$  &  0  &  $b$\\
				\hline
			\end{tabular}
			\vspace{10pt}\\
			\begin{tabular}[c]{c | c c | c c }
				%\multicolumn{2}{c}{} \vspace{7pt}\\
				\centering
				% & \\
				%\hline
				&$B_1$&&$B_2$\\
				\hline
				$A_1$&$\frac{\epsilon}{2}$, & $\frac{\epsilon}{2}$  &  
				$\frac{\epsilon^2 + b^2}{b+\epsilon}$, & 0\\[2pt]
				\hline
				$A_2$&0, & $\frac{\epsilon^2 + b^2}{b+\epsilon}$  &  $\frac{b}{2}$, & $\frac{b}{2}$\\[3pt]
				%\hline
			\end{tabular}
		\end{center}	
	\end{table}
	\begin{itemize}
		\item PoA $= \frac{b}{\epsilon}$.
	\end{itemize}
\end{frame}


\begin{frame}
	\frametitle{Summary of Our Results +(PoA)}
	\begin{table}[ht]
		\begin{center}
			\renewcommand{\arraystretch}{1.25}
			\begin{tabular}{ r|c|c|c| }
				\multicolumn{1}{r}{}
				&  \multicolumn{1}{r}{Linear Link}
				&  \multicolumn{1}{r}{Bradley-Terry}
				&  \multicolumn{1}{r}{Softmax}\\
				\cline{2-4}
				PNE w/ egoism & \checkmark & $\times$ & \checkmark \\
				%\cline{2-4}
				PNE w/o egoism & $\times$ & $\times$ & ?$^{\#}$  \\
				\cline{2-4}
				\hline
				\hline
				PoA upper bound w/ egoism & $2$ & $2$ & $1+e$ \\
				PoA lower bound w/ egoism & $2$ & $6/5$ & $2$ \\
				Worst PoA w/o egoism & $\infty$ & $\infty$ & $\infty$ \\
				\cline{2-4}
			\end{tabular}
		\end{center}
	\end{table}
\end{frame}


%%%%%%%%%%%%%%%%%%%%%%%%%%%%%%%%%%%%%%%%%%%%%%%%%%%%%%%%%%%%%%%%%%%%%
\section{Concluding Remarks}
%%%%%%%%%%%%%%%%%%%%%%%%%%%%%%%%%%%%%%%%%%%%%%%%%%%%%%%%%%%%%%%%%%%%%

\begin{frame}
\frametitle{Outline}
\tableofcontents[currentsection,currentsubsection]
\end{frame}


\begin{frame}{Future Work}
	\begin{table}[ht]
		\begin{center}
			\begin{tabular}{ r|c|c|c| }
				\multicolumn{1}{r}{}
				&  \multicolumn{1}{r}{Linear Link}
				&  \multicolumn{1}{r}{Bradley-Terry}
				&  \multicolumn{1}{r}{Softmax}\\
				\cline{2-4}
				PNE w/ egoism & \checkmark & $\times$ & \checkmark \\
				%\cline{2-4}
				PNE w/o egoism & $\times$ & $\times$ & \alert{?$^{\#}$}  \\
				\cline{2-4}
				\hline
				\hline
				PoA upper bound w/ egoism & $2$ & $2$ & \alert{$1+e$} \\
				PoA lower bound w/ egoism & $2$ & \alert{$6/5$} & \alert{$2$} \\
				Worst PoA w/o egoism & $\infty$ & $\infty$ & $\infty$ \\
				\cline{2-4}
			\end{tabular}
		\end{center}
	\end{table}
\end{frame}


\begin{frame}{Future Work (contd.)}
\begin{itemize}
	\item Three or more parties.
		\begin{itemize}
			\item How to define the winning probabilities?
		\end{itemize}
	\vspace{6pt}
	\pause
	\item The correspondence between macro and micro settings.
	\vspace{6pt}
	\pause
	\item More general models.
	\begin{itemize}
		\item Extension to monotone game.
	\end{itemize}
	\vspace{5pt}
	\item PoA w.r.t. NE.
\end{itemize}	
\end{frame}


\begin{frame}{Future Work (contd.)}
	\begin{itemize}
		\item Election campaign $\rightarrow$ Project proposal. 
		\vspace{6pt}
		\item Winner-takes-all $\rightarrow$ Budget or prize shared in proportion. 
	\end{itemize}	
\end{frame}


\iffalse
\begin{frame}
\begin{figure}
	\begin{center}
		\includegraphics[scale=0.45]{eps/handwritten.eps}
	\end{center}
\end{figure}
\end{frame}    
\fi


%%%%%%%%%%%%%%%%%%%%%%%%%%%%%%%%%%%%%%%%%%%%%%%%%%%%%%%%%%%%%%%%%%%%%
\section*{}
%%%%%%%%%%%%%%%%%%%%%%%%%%%%%%%%%%%%%%%%%%%%%%%%%%%%%%%%%%%%%%%%%%%%%


\begin{frame}
\frametitle{}
\vspace{1.8cm}
\begin{center}
{\Huge Thank you.}
\end{center}
\vspace{1.0cm}
\begin{scriptsize}
\begin{center}
\begin{itemize}
\item [] *Special Acknowledgment: Inserted Pictures Were Designed by Freepik.
\end{itemize}
\end{center}
\end{scriptsize}
\end{frame}




\end{document}


