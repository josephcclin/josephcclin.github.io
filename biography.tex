\documentclass[10pt,a4paper]{article}
\textheight=25.0cm \textwidth=16.0cm \voffset=-2.8cm \hoffset=-1.0cm
\usepackage{fontspec}
\usepackage[boldfont]{xeCJK} 
\setCJKmainfont[BoldFont=AR PL UKai TW]{AR PL UMing TW MBE} 
\XeTeXlinebreaklocale "zh"  
\XeTeXlinebreakskip = 0pt plus 1pt %這兩行一定要加,中文才能自動換行
\evensidemargin=1.0cm \oddsidemargin=1.0cm
\usepackage{booktabs,amssymb,parskip}
\usepackage{natbib}
\usepackage{setspace}
\usepackage{amsthm,bm}
%\usepackage{multirow}
\usepackage{enumerate}
\usepackage{varioref}
%\usepackage{charter,eulervm}
%\usepackage{euler}
\usepackage{amsmath}
\usepackage{latexsym}
\usepackage{amssymb}
\usepackage{graphicx}
\usepackage{amsthm}
\usepackage{comment}
\usepackage{boxedminipage}
\usepackage[small]{titlesec}
\usepackage{fancyhdr,array,tabularx,latexsym,psfrag}
\usepackage{titlesec}
\usepackage{booktabs}
\newcommand{\myparbox}[2]{\hangafter=1 \settowidth{\hangindent}{{#1} }{#1} #2}
\newcommand{\DNA}{\mathsf{D}\mspace{-2mu}\mathsf{N}\mspace{-2mu}\mathsf{A}}
\newcommand{\ora}[1]{\ensuremath{\overrightarrow{#1}}}
\newcommand{\ola}[1]{\ensuremath{\overleftarrow{#1}}}
\newcommand{\orla}[1]{\ensuremath{\overleftrightarrow{#1}}}


\begin{document}
%\linespread{1.5}
\onehalfspacing

\title{\bf 自傳}
\author{\large {\bf 林莊傑} (Joseph Chuang-Chieh Lin, 2017)}
\date{}

\maketitle

%\tableofcontents


%%%%%%%%%%%%%%%%%%%%%%%%%%%%%%%%%%%%%%%%%%%%%%%%%%%%%%%%%%%%%%%%%%%
\section*{\bf\large 一、家庭背景與求學歷程}
\label{sec:personal-information}
%%%%%%%%%%%%%%%%%%%%%%%%%%%%%%%%%%%%%%%%%%%%%%%%%%%%%%%%%%%%%%%%%%%

\pagenumbering{arabic}
\setcounter{page}{1}


我的家鄉在台南市,家中排行老大,有一個弟弟。結婚將滿十年,育有兩女。
目前一家四口居住於新北市汐止區。妻子為全職家庭主婦,
我則是在中央研究院資訊科學研究所擔任博士後研究人員。
感謝上帝,我們一家人的生活幸福美滿,和樂融融。


從小我對數學就特別有興趣,非常喜歡專心思考一些數學問題,
而且可以長時間專注在一個問題上。
小學時代,記得有一次老師出了一道數學題目,要我們算出 $1+2+3+\cdots + 
99 + 100$ 的答案,我直覺有一個方法可以快速得到答案,
而非土法煉鋼地使用 99 次加法運算求解,於是在思索了兩個小時之後,
總算找到一個正確又有效率的方法將作業完成。
過了好幾年以後我才知道,原來大數學家高斯 (Gauss, Carl Friedrich) 
小時候曾經也用過同樣的方法解決這個問題,不過他所花費的時間當然遠小於兩個小時。


國中畢業後,我通過保送甄試進入台南一中的數理資優班就讀,
一個學期過後回到普通班。
大學聯考放榜後,選擇進入家鄉的成功大學數學系就讀。
大學四年級時,發現自己真正的興趣是計算機科學,
於是我下定決心繼續升學,全力準備計算機科學相關科系的研究所考試。
在考取暨南國際大學的資工所碩士班之後,
因為對演算法設計分析有濃厚興趣,遂拜入李家同教授門下。


就讀研究所碩士班的日子裡,我修習許多關於計算理論與演算法的課。
在研究所的高等演算法這門課,
我的學期成績高達 99 分。這樣的成績深深激勵著我,
加上李教授的鼓勵,繼續升學的決定便悄悄地在內心萌芽。
經由李教授的推薦,我在拿到碩士學位後繼續到中正大學資訊工程系進修博士班,
由張貿翔教授指導,進行演算法相關問題的研究。


博士班的第一年,我順利通過系上的資格考試,成為博士候選人。
之後花了兩年左右的時間找尋適合自己的研究題目,後來聚焦在隨機演算法 
(randomized algorithms) 相關領域上。
在 2008 年,我獲得台德三明治計畫獎學金,
前往德國 RWTH Aachen University 與 
Peter Rossmanith 教授進行為期一年的合作研究。
在這一年當中,我的研究工作有很大的進展,博士論文的內容也有了基本構想。
此外,因為接受了豐富的文化刺激,我對整個世界的觀感徹底改變。
現在,我對於「世界公民」這四個字,有了更深的體會。



%%%%%%%%%%%%%%%%%%%%%%%%%%%%%%%%%%%%%%%%%%%%%%%%%%%%%%%%%%%%%%%%%%%
\section*{\bf\large 二、研究專長與得獎}
\label{sec:personal-background}
%%%%%%%%%%%%%%%%%%%%%%%%%%%%%%%%%%%%%%%%%%%%%%%%%%%%%%%%%%%%%%%%%%%


隨機演算法與固定參數演算法的設計與分析是我主要的研究專長。  
雖然在台灣絕大多數的計算機科學的學者並不熟悉這兩個領域, 
但是, 最近二十年內它們蓬勃發展, 吸引了許多歐美頂尖的理論計算機科學家投入研究。
固定參數演算法在實務上也可以找到許多應用, 
像是在生物資訊 (bioinformatics) 上就有很多可以探討的問題。
另一方面, 隨機演算法領域當中我特別專注於性質測試 (property testing) 這部份。 
性質測試的工作要求只從輸入資料中抽樣一小部份來觀察, 
就要能夠回答該筆輸入資料是否具有特定性質, 或者必須修改相當大的比例才會擁有這個性質。 
在真實世界裡, 輸入的資料量經常是很龐大的, 所以性質測試是一個很不錯的切入點。


我的博士論文裡,針對四元樹一致性 (tree-likeness of quartet topologies) 
這個與演化樹重建有關的生物計算問題進行探討,
分別提出了有效率的固定參數演算法與次線性時間 (sublinear time) 的隨機演算法。
除此之外,亦探討結合隨機演算法與固定參數演算法這兩種研究方法的可能性。
猶記得某日早晨幫女兒沖泡配方奶、洗奶瓶時,突然靈光乍現,
想出了拓展先前研究成果的方法,便振筆疾書,在一週內完成草稿準備投往期刊發表,
後來這結果也順利被 Information Processing Letters 接受刊登。

除了自己本身的研究以外, 我亦幫忙張教授指導學弟妹的碩士論文。 
在 2009 年指導碩士班學妹的碩士論文, 
該論文經編寫後投稿至嘉義大學舉辦的組合數學與計算理論研討會, 
榮獲大會的最佳論文獎 (Best Paper Award)。

2011 年至 2014 年,我在中央研究院基因體中心從事生物資訊相關研究, 
並與實驗室主持人莊樹諄研究員合作開發出 ExonFinder 套件, 
可利用跨物種的遺傳序列分析,找出具有高可信度的新穎外顯子 (novel exons) 
且附帶演化速率分析。 
該套件是以 C 語言搭配 shell scripts (Bash \& AWK) 開發完成, 
並能在 Linux 環境下編譯執行,不僅可信度高,執行效率也高。
2014 年秋天,我投入資訊所呂及人研究員的實驗室, 
進行賽局理論與演算法相關問題的研究合作。
目前對於用在開放式課程的同儕評分機制與應用、 
以及政黨政治的賽局均衡點之社會代價分析均有初步成果。


我的研究專長大致涵蓋以下領域:  演算法設計與分析, 生物資訊, 賽局理論。 
我的發表著作有五篇 SCI 期刊論文, 兩篇國際研討會與一篇國內研討會論文,
 可以參考我的個人簡歷。


%%%%%%%%%%%%%%%%%%%%%%%%%%%%%%%%%%%%%%%%%%%%%%%%%%%%%%%%%%%%%%%%%%%
\section*{\bf\large 三、興趣嗜好}
\label{sec:hobbies}
%%%%%%%%%%%%%%%%%%%%%%%%%%%%%%%%%%%%%%%%%%%%%%%%%%%%%%%%%%%%%%%%%%%


剛進入大學的我,慢慢養成運動的習慣。喜歡打籃球和慢跑。
後來田徑代表隊招募新生,受學長邀請進入學校的田徑代表校隊一起跑步。
因為跑不快、跳不高,弱不禁風的我更沒辦法碰鉛球鐵餅,於是選定中長距離項目來練習。
大學三年級時,我擔任成大田徑代表校隊的中長部長,
負責監督隊上中長距離項目的學弟妹練習並給予指導。
個人的主要參賽項目為 800 公尺與 1500 公尺徑賽,
曾在 2001 年全國大專運動會乙組的 800 公尺徑賽項目獲得第五名。
在我就讀過的三間大專院校 (成功大學、暨南大學、中正大學) 我都拿過校園越野賽的第一名。
2010 年更以 31 歲的年紀,打破兩項中正大學校運紀錄 (1500 公尺與 5000公尺;
5000 公尺原紀錄保持人為超馬好手林義傑)。 


就讀研究所以後,我開始接觸長跑運動,挑戰 42.195 公里的全程馬拉松長跑。
在德國的一年生活中,
也曾和德國好友 Josef Kunze 交換穿上印有國旗的衣服完成杜塞朵夫全程馬拉松 
(D\"{u}sseldorf Marathon),利用跑步達到國民外交的效果。
在 2009 年台南古都馬拉松、 2014 年台北國道馬拉松、 2015 年日本神戶馬拉松、 
2017 年日本京都馬拉松與新北市萬金石馬拉松, 我五度跑進三小時大關 (最佳成績 2:54:10), 
實現業餘跑者的夢想。 跑步不僅是興趣, 也激勵著我, 
只要堅持下去, 沒有不可能的事情。


中學時代我是弱不禁風的書生,利用慢跑鍛鍊自己的身體,
之後竟能在徑賽跑道上為學校爭光,在異國的土地上讓世界看見台灣;
甚至在我邁入 30 歲後,仍舊創造出許多看似不可能達成的紀錄。
跑步不僅是興趣,也激勵著我,只要堅持下去,沒有不可能的事情。



%%%%%%%%%%%%%%%%%%%%%%%%%%%%%%%%%%%%%%%%%%%%%%%%%%%%%%%%%%%%%%%%%%%
\section*{\bf\large 四、結語}
\label{sec:prospection}
%%%%%%%%%%%%%%%%%%%%%%%%%%%%%%%%%%%%%%%%%%%%%%%%%%%%%%%%%%%%%%%%%%%


``Impossible Is Nothing." 我總是這樣勉勵著自己。 
不管在研究上或是在跑步競賽上, 總是披荊斬棘, 終能歡呼收割。 
取得博士學位之後, 面臨更多的挑戰, 相信人生也因此變得更加精彩。 
我秉持著 「件件工作反映自我, 凡經我手必為佳作」 的精神, 
專注把事情作到最好。 我有自信能勝任各種工作。




\begin{comment}
%%%%%%%%%%%%%%%%%%%%%%%%%%%%%%%%%%%%%%%%%%%%%%%%%%%%%%%%%%%%%%%%%%%%%
\section{參考文獻}
\label{sec:references}
%%%%%%%%%%%%%%%%%%%%%%%%%%%%%%%%%%%%%%%%%%%%%%%%%%%%%%%%%%%%%%%%%%%%%
\vspace{-1.0cm}
\begin{thebibliography}{50}


\end{thebibliography}
\end{comment}

\end{document}
